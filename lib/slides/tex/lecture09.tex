
\begin{frame}
  \LectureNo{11}
  \maketitle
\end{frame}

\begin{frame}{Overview}
  \setcounter{framenumber}{249}
  \tableofcontents
\end{frame}

\section{Rehash}
\begin{frame}{Rehash}
	
	\begin{itemize}%[<+->]
	\itemsep=16pt
	
		\item We analyze the inner structure of atomics in f.o. logic.
		
		\item  Terms stand for objects, predicates for properties/relations.
		
		\item Quantifiers allow us talk about things in generality.
		
		\item A formula without free variables is closed/a sentence.
							
	\end{itemize}

\end{frame}
		

\section{9. Semantics for First-Order Logic}
\subsection{9.1 Truth, Models, and Assignments}

\begin{frame}{9.1 Truth, Models, and Assignments}

	\begin{itemize}%[<+->]
	\itemsep=16pt
		
		\item Our aim is to define models, which are structures that interprets the non-logical vocabulary of our formal language.
		
		\begin{itemize}
			
			\item in propositional logic, that meant sentence letters
			
			\item now, in f.o. logic, it means the entire signature
		
		\end{itemize}
		
	\end{itemize}

\end{frame}

\begin{frame}{Towards Models}

	\begin{itemize}%[<+->]
	\itemsep=16pt
	
		\item Names denote objects, predicates apply to sets (of sequences) of objects that have a certain property or relation, e.g.
		
		\begin{itemize}
		
			\item $a$ denotes the ball
				
			\item $P$ expresses $\{x:\text{x is red}\}$
			
			\item then, $P(a)$ is true iff the ball is a member of $\{x:\text{x is red}\}$ 
		
		\end{itemize}		
		
		\item Roughly, in any model $\mathcal{M}$ we have: 
	
			\begin{itemize}
			
				\item each $t$ denotes an object, indicated by $\llbracket t\rrbracket^\mathcal{M}$
				
				\item each $R$ expresses a set, indicated by $R^\mathcal{M}$
				
				\item $R(t_1, \mathellipsis, t_n)$ is true in $\mathcal{M}$ iff $(\llbracket t_1\rrbracket^\mathcal{M}, \mathellipsis, \llbracket t_n\rrbracket^\mathcal{M})\in R^\mathcal{M}$
				
			\end{itemize}

	\end{itemize}

\end{frame}

\begin{frame}{Denotations of Terms}

	(9.1.7) However, different terms get different denotations.
	
	\medskip
		
	\begin{itemize}%[<+->]
	\itemsep=16pt
			
				\item Each constant simply denotes an object.
				
				\begin{itemize}
				
					\item[] $a\in\mathcal{C}$ denotes an object indicated by $a^\mathcal{M}$
					\item[] naturally, $\llbracket a\rrbracket^\mathcal{M}=a^\mathcal{M}$ for any constant
				
				\end{itemize}
				
				\item Each function symbol $f^n\in\mathcal{F}$ expresses a function $f^\mathcal{M}$ that maps $n$-tuples of objects to to some object. The denotation of a \emph{function term} is calculated recursively, e.g.
				\small
				\[
				\llbracket f(a,f(b,c))\rrbracket^\mathcal{M}
				\;=\;
				f^\mathcal{M}(a^\mathcal{M}, \llbracket f(b,c)\rrbracket^\mathcal{M})
				\;=\;
				f^\mathcal{M}(a^\mathcal{M}, f^\mathcal{M}(b^\mathcal{M}, c^\mathcal{M}))
				\]
				\normalsize
	
				\item Constant/Function denotations are fixed and unchanging.

			\end{itemize}

\end{frame}

\begin{frame}{And the Variables?}

	\begin{itemize}%[<+->]
	\itemsep=16pt
	
	\item Variables are a bit different. We don't want them to have totally fixed denotation, like pronouns  ``he,'' ``she,'' ``it'', ``that''.
	
		\item We introduce \emph{assignments} $\alpha$ on top of a model, which give a `temporary, changeable' denotation to each variable $x$, then:
		
		\begin{itemize}
		
			\item[] $\llbracket x\rrbracket^\mathcal{M}_\alpha=\alpha(x)$
			\quad
			\alert{(note the relativization to subscript $\alpha$)}
		
		\end{itemize}
	
	\end{itemize}


\end{frame}

\begin{frame}{Truth}

  \small
	\begin{itemize}%[<+->]
	\itemsep=8pt
		
		\item Given \emph{just} the denotations determined by model $\mathcal{M}$ and assignment $\alpha$, we can calculate a truth-value for any formula.
		
		\medskip
		
		\begin{itemize}
		\itemsep=10pt
		
			 \item { $\llbracket R(t_1, \mathellipsis, t_n)\rrbracket^\mathcal{M}_\alpha=1$ iff $(\llbracket t_1\rrbracket^\mathcal{M}_\alpha, \mathellipsis, \llbracket t_n\rrbracket^\mathcal{M}_\alpha)\in R^\mathcal{M}$}
				
			\item {$\llbracket t_1=t_2\rrbracket^\mathcal{M}_\alpha=1$  iff $\llbracket t_1\rrbracket^\mathcal{M}_\alpha=\llbracket t_2\rrbracket^\mathcal{M}_\alpha$ \quad \alert{(special rule for `$=$')}}
			
			\item We apply truth-functions as before.
				\begin{itemize}
			
				\item $\llbracket \neg\phi\rrbracket^\mathcal{M}_\alpha=f_\neg(\llbracket\phi\rrbracket^\mathcal{M}_\alpha)$	
			
				\item $\llbracket\phi\circ\psi\rrbracket^\mathcal{M}_\alpha=f_\circ(\llbracket\phi\rrbracket^\mathcal{M}_\alpha,\llbracket\psi\rrbracket^\mathcal{M}_\alpha)$
			
				\end{itemize}
					
			\item But what about quantified formulas? Idea: we survey changes from assignment $\alpha$ to $\alpha[x\mapsto d]$ which is the assignment \emph{like $\alpha$} except that it changes the value of $x$ to denote object $d$.

				\begin{itemize}
				
				\item $\forall x\phi$ is true under $\mathcal{M}$ and $\alpha$ \,iff\, for every relevant object $d$, the formula $\phi$ is true under the change of assignment $\alpha[x\mapsto d]$ 
				\item $\exists x\phi$ is true under $\mathcal{M}$ and $\alpha$ \,iff\, for at least one relevant object $d$, the formula $\phi$ is true under change of assignment $\alpha[x\mapsto d]$ 
		
				\end{itemize}
				
		\end{itemize}
		
	\end{itemize}

\end{frame}

\begin{frame}{Domains}

	\begin{itemize}%[<+->]
	\itemsep=16pt
	
		\item However, to make this as clear and precise as possible, we do not assume that are working with `absolutely all' objects.
		
		\item Instead, each model $\mathcal{M}$ comes equipped with a domain $D^\mathcal{M}$.
		
		\item Think of it this way: a model is a reasoning situation and the domain is what we are reasoning about; when we use the quantifiers `something' or `everything' in such a situation we are generalization \emph{about things in the current domain}.
			
	\end{itemize}

\end{frame}

\subsection{9.2 Models and Assignments}

\begin{frame}{9.2 Official Definition of Models and Assignments}

	\begin{itemize}%[<+->]
	\itemsep=16pt
		
		\item (9.2.1) A \emph{model} for $\mathcal{S}$ is a structure $\mathcal{M}=(D^\mathcal{M},\cdot^\mathcal{M})$ where:
		\begin{enumerate}[(i)]
		
			\item $D^\mathcal{M}$ is a non-empty set
			
			\item $\cdot^\mathcal{M}$ is an \emph{interpretation function} that does these jobs:
			\begin{enumerate}[(a)]
			
				\item for every $c\in\mathcal{C}$, it gives an object $c^\mathcal{M}\in D^\mathcal{M}$.

				\item for every $f^n\in\mathcal{F}$, it gives a function $f^\mathcal{M}:D^n\to D$

				\item for every $R^n\in\mathcal{R}$, it gives a set $R^\mathcal{M}\subseteq D^n$.
			
			\end{enumerate}
		\end{enumerate}

		\item (9.2.4) An \emph{assignment} in a model $\mathcal{M}=(D^\mathcal{M},\cdot^\mathcal{M})$ of signature $\mathcal{S}$ is a function $\alpha:\mathcal{V}\to D^\mathcal{M}$. 

	\end{itemize}

\end{frame}

\begin{frame}{Examples (9.2.2)}

Models of signature $\mathcal{S}_{PA}=(\{0\}, \{S^1, +^2, \cdot^2\}, \emptyset)$:

\bigskip
				
				\begin{enumerate}[(a)]
				\itemsep=16pt
				
					\item The standard, intended model:
					
						\begin{itemize}
					
							\item $D^\mathcal{M}=\mathbb{N}$
							
							\item $0^\mathcal{M}=0$
					
							\item $S^\mathcal{M}(n)=n+1$
							
							\item $+^\mathcal{M}(n,m)=n+m$
							
							\item $\cdot^\mathcal{M}(n,m)=n\cdot m$
					
						\end{itemize}
						
					\item A natural, but non-intended model on the even numbers:
					
						\begin{itemize}
					
							\item $D^\mathcal{M}=\{n\in\mathbb{N}:n\text{ is even}\}$
																					\item $0^\mathcal{M}=0$
					
							\item $S^\mathcal{M}(n)=n+2$
							
							\item $+^\mathcal{M}(n,m)=n+m$
							
							\item $\cdot^\mathcal{M}(n,m)=n\cdot m$
					
						\end{itemize}
\end{enumerate}

\end{frame}

\begin{frame}{Examples (9.2.5)}

With assignments only the domain counts:

\bigskip

			\begin{enumerate}[(a)]
			\itemsep=16pt
		
			\item Some assignments when the domain is $D^\mathcal{M}=\mathbb{N}$
			
				\begin{enumerate}[(a)]
				
					\item $\alpha(x_1)=0, \alpha(y)=1, \alpha(z)=2$
					
					\item $\alpha(x)=1, \alpha(y)=0, \alpha(z)=3$
					
					\item $\alpha(x)=0, \alpha(y)=0, \alpha(z)=0$
					
					\item $\alpha(x)=1$ for all $x\in\mathcal{V}$
									
				\end{enumerate}
				
			\item On the other hand, for $D^\mathcal{M}=\{\ast\}$ there is \emph{one} assignment (period), viz. the constant assignment $\alpha(x)=\ast$ for all $x$.

		\end{enumerate}

\end{frame}


\begin{frame}{9.2 Official Definition of Denotations}

	\begin{itemize}%[<+->]
	\itemsep=16pt
					
		\item (9.2.6) We define the \emph{denotation} $\llbracket t\rrbracket_\alpha^\mathcal{M}$ of a term $t$ in a model $\mathcal{M}$ under assignment $\alpha$ by the following recursion:
		
			\begin{enumerate}[(a)]
			
				\item		\begin{enumerate}[(i)]

					\item $\llbracket x\rrbracket_\alpha^\mathcal{M}=\alpha(x)$
					\item $\llbracket c\rrbracket_\alpha^\mathcal{M}=c^\mathcal{M}$
				
				\end{enumerate}
				
				\item $\llbracket f(t_1,\mathellipsis,t_n)\rrbracket_\alpha^\mathcal{M}=f^\mathcal{M}(\llbracket t_1\rrbracket_\alpha^\mathcal{M}, \mathellipsis, \llbracket t_n\rrbracket_\alpha^\mathcal{M})$
			
			\end{enumerate}
			
			\item Change of assignment $\alpha[x\mapsto d]$ is defined by:
		\[\alpha[x\mapsto d](y)=\begin{cases} y &\text{if }y\neq x\\ d & \text{if }y=x\end{cases}\]		

	
	\end{itemize}

\end{frame}

\begin{frame}{Examples (9.2.7)}

\begin{enumerate}[(i)]
\itemsep=16pt
		
			\item Take the standard model for $\mathcal{S}_{PA}$ and consider any assignment that assigns $\alpha(x)=0,\alpha(y)=1,\alpha(z)=2$, then we have:
			\begin{align*}
			\llbracket 0\rrbracket^\mathcal{M}_\alpha&=0\\
			\llbracket x\rrbracket^\mathcal{M}_\alpha&=0\\
			\llbracket S(0)\rrbracket^\mathcal{M}_\alpha&=1\\
			\llbracket y\cdot S(0)\rrbracket^\mathcal{M}_\alpha&=1\\
			\llbracket S(((x\cdot y)+z))\rrbracket^\mathcal{M}_\alpha&=3
			\end{align*}
			
			
\end{enumerate}
\end{frame}
\begin{frame}{Examples (9.2.7)}

\begin{enumerate}[(i)]
\itemsep=16pt
\setcounter{enumi}{1}			
			\item In the non-intended model for $\mathcal{S}_{PA}$ (9.1.2.i.b), which has $D^\mathcal{M}=\{x:x\text{ is even}\}$, with $\alpha(x)=0,\alpha(y)=0,\alpha(z)=0$
			\begin{align*}
			\llbracket 0\rrbracket^\mathcal{M}_\alpha&=0\\
			\llbracket x\rrbracket^\mathcal{M}_\alpha&=0\\
			\llbracket S(0)\rrbracket^\mathcal{M}_\alpha&=2\\
			\llbracket y\cdot S(0)\rrbracket^\mathcal{M}_\alpha&=0\\
			\llbracket S(((x\cdot y)+z))\rrbracket^\mathcal{M}_\alpha&=2
			\end{align*}
			
	\end{enumerate}
\end{frame}

\begin{frame}{The Term Locality Lemma}

 (9.2.9) \textbf{Lemma}. Let $\mathcal{M}$ be a model and $t$ a term with precisely the variables in set $V$ in it. Then for all valuations $\alpha$ and $\beta$ in $\mathcal{M}$, if $\alpha(x)=\beta(x)$ for all $x\in V$, then $\llbracket t\rrbracket_\alpha^\mathcal{M}=\llbracket t\rrbracket_\beta^\mathcal{M}$. (proof by induction)
		
\medskip

\begin{itemize}
\itemsep=16pt

	\item Induction base:
	
	\medskip
	
	\begin{itemize}
	\itemsep=10pt
	
	\item If $t$ is a constant $a\in\mathcal{C}$, we can reason that $\llbracket a\rrbracket^\mathcal{M}_\alpha=a^\mathcal{M}=\llbracket a\rrbracket^\mathcal{M}_\beta$ for all assignments $\alpha,\beta$. 
	
	\item If $t$ is a variable $x$, then $V=\{x\}$ and so $\llbracket x\rrbracket^\mathcal{M}_\alpha=\alpha(x)=\beta(x)=\llbracket a\rrbracket^\mathcal{M}_\beta$. 
	
	\end{itemize}
	
	\item Induction step:

	\end{itemize}
	\itemsep=10pt

	{\small
	\begin{center}
		\begin{tabular}{c c c c c c ll}
		$\llbracket f(t_1, \mathellipsis, t_n)\rrbracket^\mathcal{M}_\alpha$ = & $f^\mathcal{M}($ & $\llbracket t_1\rrbracket^\mathcal{M}_\alpha$ &  \dots & $\llbracket t_n\rrbracket^\mathcal{M}_\alpha)$\\
		 & & \rotatebox{90}{=} & & \rotatebox{90}{=} &(I.H.)\\
		& $f^\mathcal{M}($ & $\llbracket t_1\rrbracket^\mathcal{M}_\beta$ &  \dots & $\llbracket t_n\rrbracket^\mathcal{M}_\beta)$&=$\llbracket f(t_1, \mathellipsis, t_n)\rrbracket^\mathcal{M}_\beta$ \\
		\end{tabular}
		\end{center}}


\end{frame}

\subsection{9.3 Truth in a Model}

\begin{frame}{9.3 Official Definition of Truth in a Model}

(9.3.1) We define the truth-value $\llbracket\phi\rrbracket^\mathcal{M}_\alpha$ of a formula $\phi$ under an assignment $\alpha$ in a model $\mathcal{M}$ by the following recursion:
		
		\medskip
		
		\begin{enumerate}[(i)]
		\itemsep=16pt
			
				\item	\begin{enumerate}[(a)]
						\itemsep=10pt

						\item $\llbracket R(t_1,\mathellipsis, t_n)\rrbracket_\alpha^\mathcal{M}=\begin{cases} 1 & \text{if }(\llbracket t_1\rrbracket^\mathcal{M}_\alpha,\mathellipsis, \llbracket t_1\rrbracket^\mathcal{M}_\alpha)\in R^\mathcal{M}\\0 &\text{otherwise}\end{cases}$
						\item $\llbracket t_1=t_2\rrbracket_\alpha^\mathcal{M}=\begin{cases} 1 & \text{if }\llbracket t_1\rrbracket^\mathcal{M}_\alpha=\llbracket t_1\rrbracket^\mathcal{M}_\alpha)\\0 &\text{otherwise}\end{cases}$				
						\end{enumerate}
				
				\item 	\begin{enumerate}[(a)]
						\itemsep=10pt

						\item  $\llbracket\neg \phi\rrbracket_v=f_\neg(\llbracket\phi\rrbracket_v)$
				
						\item  $\llbracket(\phi\circ \psi)\rrbracket_v=f_\circ( \llbracket\phi\rrbracket_v, \llbracket\psi\rrbracket_v)$ for $\circ=\land,\lor,\to,\leftrightarrow$
				
						\item $\llbracket\exists x\phi\rrbracket_\alpha^\mathcal{M}=max(\{\llbracket \phi\rrbracket^\mathcal{M}_{\alpha[x\mapsto d]}: d\in D^\mathcal{M}\})$
				
						\item[] $\llbracket\forall x\phi\rrbracket_\alpha^\mathcal{M}=min(\{\llbracket \phi\rrbracket^\mathcal{M}_{\alpha[x\mapsto d]}: d\in D^\mathcal{M}\})$
								
						\end{enumerate}
			
			\end{enumerate}

\end{frame}

\begin{frame}{Truth as a Property}

	\begin{itemize}%[<+->]
	\itemsep=16pt
	
		\item We can also write $\mathcal{M},\alpha\vDash\phi$ to mean $\llbracket\phi\rrbracket^\mathcal{M}_\alpha=1$.
		
		\item Then, for model $\mathcal{M}$ and assignment $\alpha$, we have:

			\medskip
			
			\begin{enumerate}[(i)]
			\itemsep=6pt
			
				\item $\mathcal{M},\alpha\vDash R(t_1,\mathellipsis, t_n)$ iff $(\llbracket t_1\rrbracket^\mathcal{M}_\alpha,\mathellipsis, \llbracket t_1\rrbracket^\mathcal{M}_\alpha)\in R^\mathcal{M}$
				
				\item $\mathcal{M},\alpha\vDash t_1=t_2$ iff $\llbracket t_1\rrbracket^\mathcal{M}_\alpha=\llbracket t_1\rrbracket^\mathcal{M}_\alpha$								
				\item $\mathcal{M},\alpha\vDash \neg\phi$ iff $v\nvDash\phi$
					
				\item $\mathcal{M},\alpha\vDash(\phi\land\psi)$ iff $\mathcal{M},\alpha\vDash\phi$ and $\mathcal{M},\alpha\vDash\psi$
				
				\item $\mathcal{M},\alpha\vDash(\phi\lor\psi)$ iff $\mathcal{M},\alpha\vDash\phi$ or $\mathcal{M},\alpha\vDash\psi$
				
				\item $\mathcal{M},\alpha\vDash(\phi\to\psi)$ iff $v\nvDash\phi$ or $\mathcal{M},\alpha\vDash\psi$
				
				\item $\mathcal{M},\alpha\vDash(\phi\leftrightarrow\psi)$ iff   either $\mathcal{M},\alpha\vDash\phi$ and $\mathcal{M},\alpha\vDash\psi$, or $v\nvDash\phi$ and $v\nvDash\psi$.				
				\item $\mathcal{M},\alpha\vDash\exists x\phi$ iff there exists a $d\in D^\mathcal{M}$, s.t.  $\mathcal{M},{\alpha[x\mapsto d]}\vDash \phi$
				
				\item $\mathcal{M},\alpha\vDash\forall x\phi$ iff for all $d\in D^\mathcal{M}$, we have $\mathcal{M},{\alpha[x\mapsto d]}\vDash \phi$
											
			\end{enumerate}
	\end{itemize}

\end{frame}

\begin{frame}{(9.3.4.i) Examples}

In the standard model for $\mathcal{S}_{PA}$:

\begin{enumerate}[(a)]
\itemsep=16pt
			
				\item $\mathcal{M},\alpha\vDash S(0)=y$
				
				$\llbracket S(0)\rrbracket^\mathcal{M}_\alpha=S^\mathcal{M}(\llbracket 0\rrbracket^\mathcal{M}_\alpha)=S^\mathcal{M}(0)=1$ and $\llbracket y\rrbracket^\mathcal{M}_\alpha=\alpha(y)=1$.
				
				\setcounter{enumi}{5}
				
				\item $\mathcal{M},\alpha\vDash\forall x(S(x)\neq 0)$

				{We need to show that for each  $n\in D^\mathcal{M}=\mathbb{N}$, we have $\mathcal{M},\alpha[x\mapsto n]\vDash S(x)\neq 0$. So let $n\in \mathbb{N}$ be an arbitrary number. We know that  $\mathcal{M},\alpha[x\mapsto n]\vDash S(x)\neq 0$ iff $\mathcal{M},\alpha[x\mapsto n]\nvDash S(x)=0$. Assume $\mathcal{M},\alpha[x\mapsto n]\vDash S(x)=0$ for contradiction. It follows that $\llbracket S(x)\rrbracket^\mathcal{M}_{\alpha[x\mapsto n]}=\llbracket0\rrbracket^\mathcal{M}_{\alpha[x\mapsto n]}$. We have $\llbracket S(x)\rrbracket^\mathcal{M}_{\alpha[x\mapsto n]}=S^\mathcal{M}(n)=n+1$. And we have $0^\mathcal{M}=0$. So we get that $n+1=0$. But we know that there is no natural number $n\in\mathbb{N}$ s.t. $n+1=0$. So, we can conclude that $\mathcal{M},\alpha[x\mapsto n]\vDash S(x)\neq 0$, as desired.}
				
				\end{enumerate}
				
				\end{frame}

\begin{frame}{(9.3.4.ii) Examples}

For signature $\mathcal{S}=(\{a,b,c\}, \{f^1, g^2\}, \{P^1, R^2\})$:
							
						
						\begin{itemize}
						
							\item $D^\mathcal{M}=\{1,2,3,4\}$
							
							\item $a^\mathcal{M}=1, b^\mathcal{M}=3, c^\mathcal{M}=2$
						
							\item $f^\mathcal{M}(x)=x$ and $g^\mathcal{M}(x,y)=min(x,y)$
							
							\item $P^\mathcal{M}=\{1,3\}$ and $R^\mathcal{M}=\{(1,1), (1,2),(2,2) (2,3), (3,3)\}$
						
						\end{itemize}
						
						

\end{frame}

\begin{frame}{(9.3.4.ii) Examples}

	Various results in the foregoing model:
	\begin{enumerate}[(a)]
	\itemsep=10pt
			
		  	\item $\mathcal{M},\alpha\vDash P(a)$

				Simply note that $\llbracket a\rrbracket^\mathcal{M}_\alpha=a^\mathcal{M}=1$ and $1\in P^\mathcal{M}$		
					
				\item $\mathcal{M},\alpha\nvDash P(z)$
				
				Simply note that $\llbracket z\rrbracket^\mathcal{M}_\alpha=\alpha(z)=4$ and $4\notin P^\mathcal{M}$
				
				\item $\mathcal{M},\alpha\vDash R(x,x)$
				
				First,  remember that $\llbracket x\rrbracket^\mathcal{M}_\alpha=1.$ It follows that $(\llbracket x\rrbracket^\mathcal{M}_\alpha,\llbracket x\rrbracket^\mathcal{M}_\alpha)=(1,1)$ and $(1,1)\in R^\mathcal{M}$.
				

        \item $\mathcal{M},\alpha\vDash \exists y (y\neq x\land R(y,y))$
				
				\item $\mathcal{M},\alpha\vDash \forall x P(g(a,x))$
				
				\item $\mathcal{M},\alpha\vDash \forall x P(a)$. 
				
			  \item[] \ldots think about how to explain these yourself!
			
			\end{enumerate}
	\end{frame}
	
\begin{frame}{Two Lemmas}

	\begin{itemize}%[<+->]
	\itemsep=20pt

		\item (9.3.7) \textbf{Lemma.} Let $\mathcal{M}$ be a model and $\phi\in\mathcal{L}$ whose free variables form the set $V$. Then for all valuations $\alpha$ and $\beta$, if $\alpha(x)=\beta(x)$ for all $x\in V$, then $\llbracket\phi\rrbracket_\alpha^\mathcal{M}=\llbracket\phi\rrbracket_\beta^\mathcal{M}$.
		

\item (9.3.7) \textbf{Corollary.}	Let $\mathcal{M}$ be a model and $\phi\in\mathcal{L}$ a sentence (i.e. a formula with no free variables). Then for all assignments $\alpha,\beta$, we have $\llbracket\phi\rrbracket_\alpha^\mathcal{M}=\llbracket\phi\rrbracket_\beta^\mathcal{M}$.
		
		
\end{itemize}

\end{frame}

\subsection{9.4 Consequence and Validity}

\begin{frame}{9.4 Consequence and Validity}

	\begin{itemize}%[<+->]
	\itemsep=16pt
	
		\item For $\Gamma$ a set of sentences and $\phi$ a sentence, we say that $\Gamma\vDash\phi$ iff for all models $\mathcal{M}$, if $\mathcal{M}\vDash\psi$ for all $\psi\in\Gamma$, then $\mathcal{M}\vDash\phi$.
			
			\item This gives us: $\Gamma\nvDash\phi$ iff there exists a (counter)model $\mathcal{M}$, such that $\mathcal{M}\vDash\psi$ for all $\psi\in\Gamma$, but at the same time $\mathcal{M}\nvDash\phi$.
		
	\end{itemize}

\end{frame}

\begin{frame}{(9.4.2) Examples}

\begin{enumerate}[(i)]
\itemsep=16pt
		
		\item $\forall xP(x)\vDash P(a)$
		
		\item[] To see this, suppose that $\mathcal{M},\alpha\vDash\forall xP(x)$ for some arbitrary model $\mathcal{M}$ and valuation $\alpha$. It follows that for all $d\in D^\mathcal{M}$, we have $\mathcal{M},\alpha[x\mapsto d]\vDash P(x)$. But $a^\mathcal{M}\in D^\mathcal{M}$. So, if we set $d=a^\mathcal{M}$, we get $\mathcal{M},\alpha[x\mapsto a^\mathcal{M}]\vDash P(x)$. But that just means that $\alpha[x\mapsto a^\mathcal{M}](x)=a^\mathcal{M}\in P^\mathcal{M}$, from which it immediately follows that $\mathcal{M},\alpha\vDash P(a)$.
		
		\item $P(a)\vDash\exists xP(x)$
		
		\item[] To see this, suppose that $\mathcal{M},\alpha\vDash P(a)$ for some arbitrary model $\mathcal{M}$ and valuation $\alpha$. This means that $a^\mathcal{M}\in P^\mathcal{M}$. But then, we can simply set $d=a^\mathcal{M}$, and get that $\mathcal{M},\alpha[x\mapsto a^\mathcal{M}]\vDash P(x)$ and so $\mathcal{M},\alpha\vDash\exists xP(x)$, as desired.

\end{enumerate}

\end{frame}

\begin{frame}{(9.4.2) Examples}

\begin{enumerate}[(i)]
\itemsep=16pt

\setcounter{enumi}{2}

\item $\exists x(P(x)\land Q(x))\vDash \exists xP(x)\land \exists xQ(x)$

		\item[] Suppose that $\llbracket\exists x(P(x)\land Q(x))\rrbracket^\mathcal{M}_\alpha=1$. That means that we can change the value of only $x$ to $d$ such that $\llbracket P(x)\land Q(x)\rrbracket^\mathcal{M}_{\alpha[x\mapsto d]}=1$. Hence $\llbracket P(x)\rrbracket\rrbracket^\mathcal{M}_{\alpha[x\mapsto d]}=1$ and $\llbracket Q(x)\rrbracket\rrbracket^\mathcal{M}_{\alpha[x\mapsto d]}=1$. So we can change the value of only $x$ to $d$ such that $\llbracket P(x)\rrbracket\rrbracket^\mathcal{M}_{\alpha[x\mapsto d]}=1$, meaning $\llbracket\exists xP(x)\rrbracket^\mathcal{M}_\alpha=1$; and we can change the value of only $x$ to $d$ such that $\llbracket Q(x)\rrbracket\rrbracket^\mathcal{M}_{\alpha[x\mapsto d]}=1$, meaning $\llbracket\exists xQ(x)\rrbracket^\mathcal{M}_\alpha=1$. Hence $\llbracket\exists xP(x)\land \exists xQ(x)\rrbracket^\mathcal{M}_\alpha=1$, as desired.
		
		\end{enumerate}

\end{frame}

\begin{frame}{(9.4.2) Examples}

\begin{enumerate}[(i)]
\itemsep=16pt

\setcounter{enumi}{7}

\item $\exists xP(x)\land \exists x Q(x)\nvDash \exists x(P(x)\land Q(x))$ 
		
		\item[] Here's a countermodel:
		
		\begin{itemize}
		
			\item $D^\mathcal{M}=\{a,b\}$
			
			\item $P^\mathcal{M}=\{a\}$
			\item $Q^\mathcal{M}=\{b\}$
		
		\end{itemize}
		
		\end{enumerate}

\end{frame}

\begin{frame}{(9.4.3) Some Useful Quantifier Laws}

	\bigskip		
	
		\begin{enumerate}[(i)]
		\itemsep=6pt
		
		\item $\forall x\phi\vDash(\phi)[x:=t]$  where $t$ is a ground term

		\item $(\phi)[x:=t]\vDash\exists x\phi$ where $t$ is a ground term

		\item $\forall x\phi\equi \neg \exists x\neg\phi$
	
			\item $\exists x\phi\equi \neg \forall x\neg \phi$
		
			\item $\forall x\forall y\phi\equi \forall y\forall x\phi$

	\item $\exists x\exists y\phi\equi \exists y\exists x\phi$

	\item $(\forall x\phi\land \forall x\psi)\equi \forall x(\phi\land \psi)$

	\item $(\exists x\phi\lor \exists x\psi)\equi \exists x(\phi\lor \psi)$

	\item $(\phi\to \forall x\psi)\equi \forall x(\phi\to \psi)$ if $x$ is not free in $\phi$

	\item $(\phi\to \exists x\psi)\equi \exists x(\phi\to \psi)$ if $x$ is not free in $\phi$

	\item $(\forall x\phi\to \psi)\equi \exists x(\phi\to \psi)$ if $x$ is not free in $\psi$

	\item $(\exists x\phi\to \psi)\equi \forall x(\phi\to \psi)$ if $x$ is not free in $\psi$

		\end{enumerate}

\end{frame}

\begin{frame}{Deduction and ICGNS}

	\begin{itemize}%[<+->]
	\itemsep=16pt
	
		\item \textbf{Theorem} (Deduction Theorem). Let $\phi,\psi\in\mathcal{L}$ be formulas and $\Gamma\subseteq\mathcal{L}$ a set of formulas. Then the following two are equivalent:
			\begin{enumerate}[1.]
			
				\item $\Gamma\cup\{\phi\}\vDash\psi$
				
				\item $\Gamma\vDash \phi\to\psi$
			
			\end{enumerate}


		\item \textbf{Theorem} (I Can't Get No Satisfaction).
			Let $\Gamma\subseteq\mathcal{L}$ be a set of formulas and $\phi\in\mathcal{L}$ a formula. Then, the following are equivalent:
			\begin{enumerate}[1.]
			
				\item $\Gamma\vDash\phi$
				
				\item $\Gamma\cup\{\neg\phi\}$ is unsatisfiable
			
			\end{enumerate}

	
	\end{itemize}

\end{frame}

\begin{frame}{Albert, Betty, and Charles}

	\begin{itemize}%[<+->]
	\itemsep=16pt
		
		\item Consider $\mathcal{S}=(\{a,b,c\}, \emptyset, \{M^1, L^2\})$, where $M$ formalizes ``\dots is married'', $L$ formalizes ``\dots looks at \underline{\phantom{\dots}}'', $a$ formalizes ``Albert,'' $b$ formalizes ``Betty,'' and $c$ formalizes ``Charles.''
		
		\item \emph{Claim}: \[\neg M(a), M(c), L(c,b), L(b,a)\vDash \exists x\exists y(M(x)\land \neg M(y)\land L(x,y)).\]
		
		\item \emph{Proof}: 	
		
		Let $\mathcal{M}$ be a model and $\alpha$ arbitrary, such that $\llbracket M(c)\rrbracket_\alpha^\mathcal{M}=1$, $\llbracket\neg M(a)\rrbracket_\alpha^\mathcal{M}=1$, $\llbracket L(c,b)\rrbracket_\alpha^\mathcal{M}=1$, and $\llbracket L(b,a)\rrbracket_\alpha^\mathcal{M}=1$. So $a^\mathcal{M}\notin M^\mathcal{M}$, $c^\mathcal{M}\in M^\mathcal{M}$, and $( c^\mathcal{M}, b^\mathcal{M}), ( b^\mathcal{M}, a^\mathcal{M})\in L^\mathcal{M}$. We have that $\llbracket \exists x\exists y(M(x)\land \neg M(y)\land L(x,y))\rrbracket_\alpha^\mathcal{M}=1$ holds iff there are changes for $x$ to $d$ and $y$ to $d'$ such that \[\llbracket (M(x)\land \neg M(y)\land L(x,y))\rrbracket_{\alpha[x\mapsto d, y\mapsto d']}^\mathcal{M}=1.\] 
	\end{itemize}

\end{frame}

\begin{frame}{Albert, Betty, and Charles}

	\begin{itemize}%[<+->]
	\itemsep=16pt
		\item \emph{Proof continued}: 	
				
		\item[] Now, we know that either (i) $b^\mathcal{M}\in M^\mathcal{M}$ or (ii) $b^\mathcal{M}\notin M^\mathcal{M}$. 	
 \begin{itemize}
			
			\item If (i) $b^\mathcal{M}\in M^\mathcal{M}$, then we can set $d=b^\mathcal{M}$ and $d'=a^\mathcal{M}$. We'd get $d\in M^\mathcal{M}$ and so $\llbracket M(x))\rrbracket_{\alpha[x\mapsto d, y\mapsto d']}^\mathcal{M}=1$; $d'\notin M^\mathcal{M}$ and so $\llbracket\neg M(y)\rrbracket_{\alpha[x\mapsto d, y\mapsto d']}^\mathcal{M}=1$; and $( d,d')\in L^\mathcal{M}$ and so $\llbracket L(x,y))\rrbracket_{\alpha[x\mapsto d, y\mapsto d']}^\mathcal{M}=1$; giving us, $\llbracket \exists x\exists y(M(x)\land \neg M(y)\land L(x,y))\rrbracket_{\alpha[x\mapsto d, y\mapsto d']}^\mathcal{M}=1$. 
			
		\item If (ii) $b^\mathcal{M}\notin M^\mathcal{M}$, we can set $d=c^\mathcal{M}$ and $d'=b^\mathcal{M}$. In a similar way, we get 
				
		\end{itemize}

Either way, $\llbracket \exists x\exists y(M(x)\land \neg M(y)\land L(x,y))\rrbracket_{\alpha[x\mapsto d, y\mapsto d']}^\mathcal{M}=1$, which is what we wanted to show.

	\end{itemize}

\end{frame}

\begin{frame}
\begin{center}
\huge \smiley
\end{center}

\end{frame}

\begin{frame}{Core Ideas (Lecture Version)}

	\begin{itemize}%[<+->]
	\itemsep=16pt
	
	\item A model interprets the signature of an f.o. language
		
	\item An assignment for a model gives `temporary' denotations to variables, like a conversational context in natural language. 
	
	\item We recursively calculate denotations of terms in any $\mathcal{M}/\alpha$.
		
	\item We recursively calculate truth-values of formulas in any $\mathcal{M}/\alpha$.
		
	\item Validity is defined by truth-preservation across all models.
		
	\end{itemize}

\end{frame}


\begin{frame}

	\begin{center}
	{\huge\bf Thanks!}
	\end{center}

\end{frame}
