%by Maarten Burger, Alexander Apers, and Jos Zuiderwijk

\chapter{Chapter 8. Syntax for First-Order Logic}

\section*{8.9 Exercises}

	
\noindent \textbf{8.9.1}
\begin{enumerate}
    \item[(i)] \begin{enumerate}
        \item[(a)] 
	$sub(R(t_1, \dots, t_n)) = \{R(t_1, \dots, t_n)\}$ for all $R^n \in \mathcal{P}$ and \\ $t_1, \dots, t_n \in \mathcal{T}$.
        \item[(b)]
	$sub(t_1 = t_2) = \{t_1 = t_2\}$ for all $t_1, t_2 \in \mathcal{T}$.
	\end{enumerate}
	\item[(ii)] \begin{enumerate}
	    \item[(a)]
	$sub(\neg \phi) = \{\neg \phi\} \cup sub(\phi)$ for all $\phi \in \mathcal L$.
        \item[(b)]
	$sub((\phi \circ \psi)) = \{(\phi \circ \psi)\} \cup sub(\phi) \cup sub(\psi)$ for all $\phi, \psi \in \mathcal L$.
	    \item[(c)]
	$sub(Qx \phi) = \{Qx\phi\} \cup sub(\phi)$ for all $\phi \in \mathcal L$ and $Q = \forall, \exists$.\\
	\end{enumerate}
\end{enumerate}

\noindent \textbf{8.9.2} \\ 
Suppose $\phi$ is a formula and $x$ is the only free variable in $\phi$, i.e. $x$ is not bound by a quantifier. Let's consider $Qx\phi$. By definition the root of the corresponding parsing tree is the following occurence: $\langle r, Qx \rangle$. Since it's the root, there is a path to the variable $x$. Therefore $x$ now is bound by $\langle r, Qx \rangle$. Since $x$ was the only free variable, $Qx\phi$ is now closed.\\

\noindent \textbf{8.9.6} \\ No, this is not the case. Consider $\exists x P(x)$. This is a sentence since all the variables it contains are bound (the $x$ is bound by the $\exists x$). The set of sub-formulas is $sub(\exists x P(x)) = \{\exists x P(x), P(x)\}$. $P(x)$ is a sub-formula, however it is not sentences since it contains an unbound variables, i.e. the $x$.\\
\newline
\noindent \textbf{8.9.9 (i)} \\
$$(\forall x(R(x,y) \rightarrow \exists y R(y,y)))[y := x]$$
$$\forall x((R(x,y) \rightarrow \exists y R(y,y)))[y := x]$$ 
$$\forall x((R(x,y))[y := x] \rightarrow (\exists y R(y,y))[y := x])$$
$$\forall x(R((x)[y := x],(y)[y := x]) \rightarrow \exists y R(y,y))$$
$$\forall x(R(x,x) \rightarrow \exists y R(y,y))$$

\vspace{2mm}
\noindent \textbf{8.9.11 c}
\newline
Vertaalsleutel: \newline
+($x,y$): De som van $x$ en $y$\newline
$G(x,y)$: $x$ is groter dan $y$ \newline
2: het getal twee \newline
3: het getal drie \newline
4: het getal vier \newline
$G(+(2,3),4)$ \newline

\vspace{2mm}
\noindent \textbf{8.9.12 c}
\newline
Vertaalsleutel: \newline
$G(x,y)$: $x$ is groter dan $y$ \newline
3: het getal drie \newline
4: het getal vier \newline
$\forall x (G(x,4) \rightarrow G(x,3))$ \newline

\vspace{2mm}
\noindent \textbf{8.9.12 f}
\newline
$K(x,y)$: $x$ is kleiner dan $y$ \newline
3: het getal drie \newline
4: het getal vier \newline
$\forall x (K(x,3)\rightarrow K(x,4)$

\vspace{2mm}
\noindent \textbf{8.9.13 c}
\newline
$E(x)$: $x$ is een even getal \newline
$O(x)$: $x$ is een oneven getal \newline
$G(x,y)$: $x$ is groter dan $y$ \newline
$K(x,y)$: $x$ is kleiner dan $y$ \newline
$\exists x (E(x) \land \exists y (O(y) \land K(x,y) \land \exists z (O(z) \land G(y,z))))$ \newline

\vspace{2mm}
\noindent \textbf{8.9.16}
\newline
\begin{enumerate}[(a)]
\item Er is iemand die sterker is dan Peter.
\item Als een persoon sterker is dan een ander dan is de ander groter dan deze persoon.
\item Iemand die groter is dan een ander is ook sterker dan een ander.
\item Peter is groter dan niemand maar niemand is sterker dan Peter.
\item Als een persoon blij is dan is er iemand die groter is dan deze persoon.
\end{enumerate}

	
%%% Local Variables: 
%%% mode: latex
%%% TeX-master: "../../logic.tex"
%%% End: 