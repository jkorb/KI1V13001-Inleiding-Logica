\chapter{Chapter 1. Introduction}

\section*{1.6 Self-Study Questions}

\emph{Explanations}: 

	\begin{enumerate}

	\item[1.6.1] \begin{enumerate}[(a)]
	
		\item If in every situation, the premises are false, there can't be a situation in which the premises are true and the conclusion is false, i.e. the inference can't be invalid. Hence the inference is valid.
		
		\item If in some situation the premises are true and the conclusion not, the conclusion is false. By definition, this means that the inference is invalid.
		
		\item If in no situation, the premises are true and the conclusion not, then the inference can't be invalid. So it must be valid.
		
		\item If in no situation the conclusion is false, there can't be a situation in which the premises are true and the conclusion is false, i.e. the inference can't be invalid. Hence the inference is valid.
		
		\item If in every situation the conclusion is false, then it's still possible that there is no situation in which the premises are true. But if there's no situation where the premises are true, then the inference would be valid.e
		
		\item If in every situation where the conclusion is false, at least one of the premises is false, then there can't be a situation in which the premises are true and the conclusion is false. For then, all the premises would be true, but also at least one of them would be false, which can't be.
	
	\end{enumerate}
	
	\item[1.6.2] \begin{enumerate}[(a)]
	
		\item Even if there is a situation in which both premises and conclusion are false, there could still be no situation in which the premises are \emph{true} and the conclusion is false.
		
		\item There could be \emph{no} situation in which the conclusion is false. Then, trivially, in every situation where the conclusion is false, the premises are true. But, at the same time, the inference would be valid, since there  couldn't be a situation in which the premises are true and the conclusion is false, i.e. the inference can't be invalid.
		
		\item Suppose the conclusion is true in no situation. This means the conclusion is false in \emph{every} situation. But if then, there's a situation in which the premises are true, in that situation the conclusion must be false. Hence, there's a situation in which the premises are true and the conclusion is false, so the inference is invalid.
		
		\item This is just the definition of what it means for an inference to be invalid.
		
		\item It could still be that there is no situation in which the premises are true. Then, trivially, in every situation in which the premises are true, the conclusion is false. But, at the same time, there couldn't be a situation in which the premises are true and the conclusion is false, the argument would be valid.
		
		\item There could still be a situation in which the premises are true and the conclusion is false, all we're given is that there is no situation in which the premises and the conclusion are both true. This just means that in every situation in which the premises are true, the conclusion is false (see previous option).
	
	\end{enumerate}
	
	\end{enumerate}
		
\section*{1.7 Exercises}

	\begin{enumerate}
	
		\item[1.7.1] \begin{enumerate}[(a)]
		
			\item The inference is not valid. To see this, note that we can have a situation in which the premises are true and the conclusion is false. Think of a situation in which there are two whales, Moby and Dick, and one more fish, the clownfish Nemo. Moby is a blue whale and Dick is a grey whale, Nemo is orange and white. For argument's sake, suppose that all whales are fish. Surely then, all blue fish are whales, since there's only one blue fish, the whale Moby. But there's a whale, Dick, which is a fish but grey. So, Dick is not a blue fish. All in all, in the situation, the premises are true and the conclusion is false, the inference is invalid.
			
			\item The inference is valid. To see this, suppose that we're in a situation in which you didn't not miss your train. Can it be, in such a situation, that you still didn't miss your train? Well, that would mean that some statement, viz. you didn't miss your train, is both true and not true. But this is impossible. So we can't have a situation in which you didn't not miss your train but still didn't miss the train. This means the inference can't be invalid, so it has to be valid.
			
			\item The inference is valid. The case is very similar to the case with the letters in the drawer. Suppose that it's true in some situation that if you'd checked your mail, then you'd have seen my message, and you didn't see it. Can it be that you checked your mail in that situation? Well, then you'd have seen my mail and you didn't. So it can't be that, in the situation, you checked your mail. So, in every situation where the premises are true, the conclusion needs to be true as well, i.e. the inference is valid. 
			
			\item The inference is invalid. Think of a possible situation in which there are no roses at all, e.g. because a rose disease wiped them out. In such a situation, trivially, every rose would be red (can you show me a non-red rose in the situation?). But there would be no rose and certainly not a red one. So there's a possible situation in which the premises are true and the conclusion is false, the inference is invalid.
			
			\item The inference is invalid. For concreteness sake, let's suppose that ``that'' is that you jumped over the Eiffel tower. Now think of a possible situation in which you have superhuman strength and, at the same time, pigs have wings and can fly (what a beautiful world it would be). Well, in such a situation, we know that no matter whether you jumped over the Eiffel tower, pigs can fly. So certainly, \emph{if} you did it, pigs can fly (how can the statement be false?). Now suppose that in our magic land, you indeed jumped over the Eiffel tower. Then the conclusion is false---you did actually do it. So, we have a situation in which if you did it, then pigs can fly and you did it---the premises are true and the conclusion is false. 
			
			If you were to add the premise that pigs \emph{don't} fly, the argument would become valid. Check that for yourself. This shows that usual figure of speech involved here is elliptic. It's assumed, in the background, that pigs can't fly.
		
		\end{enumerate}
		
	\item[1.7.2] Suppose that an inference is valid. By definition, this means that in every situation where the premises are true, the conclusion is true as well. Suppose you add some premises to the inference. Can that inference be invalid? Well there would need to be a situation in which all the premises of the new inference are true but the conclusion is false. But all the premises of the old argument are also premises of the new inference, so in such a situation also all the premises of the old inference would need to be true. But since the old inference is valid, this means that the conclusion would be true in the situation. So for the new inference to be invalid, the conclusion would need to be both false and true in some situation, which is impossible. Hence the new inference is valid.
	
	\item[1.7.3] Take some invalid inference. Here are two ways of making it valid:
	
		\begin{enumerate}[(1)]
		
			\item Add the conclusion of the inference as a premise. It's easy to see that the new inference can't be invalid, for there would need to be a situation in which the new premises, which now include the conclusion, are all true but the conclusion is false. So the conclusion would need to be both true and false in some situation which can't be. So the inference is valid.
			
			\item Add any contradiction, such as ``the rose is red and not red'' to the premises. We've already seen that any inference with inconsistent premises is valid (1.3.3), so the inference will be valid, too.
		
		\end{enumerate}
		
	\end{enumerate}
	
%%% Local Variables: 
%%% mode: latex
%%% TeX-master: "../../logic.tex"
%%% End: 